% This file was converted to LaTeX by Writer2LaTeX ver. 1.4
% see http://writer2latex.sourceforge.net for more info
\documentclass{article}
\usepackage[utf8]{inputenc}
\usepackage[T1]{fontenc}
\usepackage[ngerman]{babel}
\usepackage{amsmath}
 \usepackage{listings}
\usepackage{amssymb,amsfonts,textcomp}
\usepackage{array}
\usepackage{color}
\usepackage{supertabular}
\usepackage{hhline}
\makeatletter
\newcommand\arraybslash{\let\\\@arraycr}
\makeatother
\setlength\tabcolsep{1mm}
\renewcommand\arraystretch{1.3}
\definecolor{zubesprechen}{RGB}{128, 0, 0}
\title{}
\begin{document}
\section{SW-Project Find-Me Platform}
Daten für datenbank:


\subsection[Users:]{Users:}
Modell:


\tablefirsthead{}
\tablehead{}
\tabletail{}
\tablelasttail{}
\begin{supertabular}{|m{5.466cm}|m{5.467cm}|m{5.467cm}|}
\hline
Feldname &
Type &
Kardinalität\\\hline
\_id &
int &
1\\\hline
login &
String &
1\\\hline
password &
String &
1\\\hline
rechte &
String(admin, normalUser, fakeUser) &
1\\\hline
\end{supertabular}


\bigskip

JSON :

\begin{flushleft}
\tablefirsthead{}
\tablehead{}
\tabletail{}
\tablelasttail{}
\begin{supertabular}{|m{3.61cm}|m{2.684cm}|}
\hline
User: &
\begin{center}
\begin{tabular}{|m{2.49cm}|}

\hline
\_id\\\hline
login\\\hline
password\\\hline
rechte\\\hline
\end{tabular}
\end{center}
\\\hline
\end{supertabular}
\end{flushleft}

So wirsd dann in JSON assehen (die eingebetteten Tabellen sind Objekte):
\begin{lstlisting}
{users: [
  {"_id":"444","login":"admin", "password": "r567t"," rechte": "admin"},
  {"_id":"555","login":"anna", "password": "56zu7", "rechte": "normalUser"},
] }
\end{lstlisting}

\newpage
\subsection[Profile:]{Profile:}
Modell:

\begin{flushleft}
\tablefirsthead{}
\tablehead{}
\tabletail{}
\tablelasttail{}
\begin{supertabular}{|m{5.466cm}|m{5.467cm}|m{3.76cm}|}
\hline
Feldname &
Type &
Kardinalität\\\hline
\_id &
int &
1\\\hline
user\_id &
int &
n (bei fake user n:1)\\\hline
Name  &
String &
1\\\hline
Vorname &
String &
1\\\hline
Geburtsdatum &
Date (Format? YYYY oder TT.MM.YYYY (8.1 Ansicht: Registrieren \ wird nur Geburtsjahr verlangt) &
1\\\hline
Geschlecht  &
String (mänlich/weiblich/sonstiges) &
1\\\hline
Registrierungsdatum &
Date &
1\\\hline
E-Mail adresse &
String &
1\\\hline
Über mich &
String &
1\\\hline
Familienstand  &
String &
1\\\hline
praeferenzen\_id &
int &
1\\\hline
freundeliste\_id &
int &
1\\\hline
freunde\_anfragen\_id &
int &
1\\\hline
\color{zubesprechen}
Kinder anzahl &
int &
1\\\hline
Haustiere &
String &
n\\\hline
Job &
String &
n\\\hline
Interessen  &
Interessen  &
1\\\hline
Aussehen &
Aussehen &
1 \color{black}\\\hline
\end{supertabular}
\end{flushleft}

\newpage
JSON:

\begin{flushleft}
\tablefirsthead{}
\tablehead{}
\tabletail{}
\tablelasttail{}
\begin{supertabular}{|m{1.2019999cm}|m{6.494cm}|}
\hline
Profil &
\begin{flushleft}
\begin{tabular}{|m{3.001cm}m{3.001cm}|}

\hline
\multicolumn{2}{|m{6.202cm}|}{\_id}\\\hline
\multicolumn{2}{|m{6.202cm}|}{user\_id}\\\hline
\multicolumn{2}{|m{6.202cm}|}{Name }\\\hline
\multicolumn{2}{|m{6.202cm}|}{Vorname}\\\hline
\multicolumn{2}{|m{6.202cm}|}{Geburtsdatum}\\\hline
\multicolumn{2}{|m{6.202cm}|}{Geschlecht }\\\hline
\multicolumn{2}{|m{6.202cm}|}{Registrierungsdatum}\\\hline
\multicolumn{2}{|m{6.202cm}|}{E-Mail adresse}\\\hline
\multicolumn{2}{|m{6.202cm}|}{Über mich}\\\hline
\multicolumn{2}{|m{6.202cm}|}{Familienstand }\\\hline
\multicolumn{2}{|m{6.202cm}|}{praeferenzen\_id}\\\hline
\multicolumn{2}{|m{6.202cm}|}{freundeliste\_id}\\\hline
\multicolumn{2}{|m{6.202cm}|}{freunde\_anfragen\_id}\\\hline
\multicolumn{2}{|m{6.202cm}|}{Kinder anzahl}\\\hline
\multicolumn{1}{|m{3.001cm}|}{Haustier} &
\begin{center}
\begin{tabular}{|m{2.807cm}|}

\hline
Tierart\\\hline
\end{tabular}
\end{center}
\\\hline
\multicolumn{2}{|m{6.202cm}|}{Job}
\\\hline
\end{tabular}
\end{flushleft}
~
\\\hline
\end{supertabular}
\end{flushleft}

Alternativ:

\begin{flushleft}
\tablefirsthead{}
\tablehead{}
\tabletail{}
\tablelasttail{}
\begin{supertabular}{|m{1.2019999cm}|m{6.494cm}|}
\hline
Profil &
\begin{flushleft}
\begin{tabular}{|m{3.001cm}m{3.001cm}|}

\hline
\multicolumn{2}{|m{6.202cm}|}{\_id}\\\hline
\multicolumn{1}{|m{3.001cm}|}{User } &
\begin{center}
\begin{tabular}{|m{2.807cm}|}

\hline
login\\\hline
password\\\hline
rechte\\\hline
\end{tabular}
\end{center}
~
\\\hline
\multicolumn{2}{|m{6.202cm}|}{Name }\\\hline
\multicolumn{2}{|m{6.202cm}|}{Vorname}\\\hline
\multicolumn{2}{|m{6.202cm}|}{Geburtsdatum}\\\hline
\multicolumn{2}{|m{6.202cm}|}{Geschlecht }\\\hline
\multicolumn{2}{|m{6.202cm}|}{Registrierungsdatum}\\\hline
\multicolumn{2}{|m{6.202cm}|}{E-Mail adresse}\\\hline
\multicolumn{2}{|m{6.202cm}|}{Über mich}\\\hline
\multicolumn{2}{|m{6.202cm}|}{Familienstand }\\\hline
\multicolumn{1}{|m{3.001cm}|}{praeferenzen} &
\begin{center}
\begin{tabular}{|m{2.807cm}|}

\hline
interessen\_id\\\hline
auserligkeiten\_id\\\hline
\end{tabular}
\end{center}
~
\\\hline
\multicolumn{1}{|m{3.001cm}|}{freund} &
\begin{center}
\begin{tabular}{|m{2.807cm}|}

\hline
profil\_id\\\hline
\end{tabular}
\end{center}
~
\\\hline
\multicolumn{1}{|m{3.001cm}|}{freunde\_anfrage} &
\begin{center}
\begin{tabular}{|m{2.807cm}|}

\hline
profil\_id\\\hline
\end{tabular}
\end{center}
~
\\\hline
\multicolumn{2}{|m{6.202cm}|}{Miderjärige Kinder anzahl}\\\hline
\multicolumn{1}{|m{3.001cm}|}{Haustier} &
\begin{center}
\begin{tabular}{|m{2.807cm}|}

\hline
Tierart\\\hline
\end{tabular}
\end{center}
~
\\\hline
\multicolumn{2}{|m{6.202cm}|}{Job}\\\hline
\end{tabular}
\end{flushleft}
~
\\\hline
\end{supertabular}
\end{flushleft}
\newpage
\subsection[Praeferenzen]{Praeferenzen}

\begin{center}
\tablefirsthead{}
\tablehead{}
\tabletail{}
\tablelasttail{}
\begin{supertabular}{|m{5.467cm}|m{5.467cm}|m{5.467cm}|}
\hline
Feldname &
Type &
Kardinalität\\\hline
\_id &
int &
1\\\hline
interessen\_id &
int &
1\\\hline
auserligkeiten\_id &
Int  &
1\\\hline
\end{supertabular}
\end{center}


\subsection[Interessen:]{Interessen:}
Interessen: (gleiche Daten müssen auch im Profil angegeben werden , sonst kann man nicht suchen)

\textcolor{zubesprechen}{(gehören Interessen jetzt zum Profil des Benutzers selbst, oder müssen noch extra
beforzugte Interessen für gesuchte Profile angegeben werden?)}

\begin{center}
\tablefirsthead{}
\tablehead{}
\tabletail{}
\tablelasttail{}
\begin{supertabular}{|m{5.467cm}|m{5.467cm}|m{5.467cm}|}
\hline
Feldname &
Type &
Kardinalität\\\hline
\_id &
int &
1\\\hline
Hobbys &
String &
n\\\hline
Bücher &
String &
n\\\hline
Filme &
String &
n\\\hline
Spiele &
String &
n\\\hline
Sport &
String &
n\\\hline
Essen &
String &
n\\\hline
Alkohol &
boolean &
1\\\hline
Rauchen &
boolean &
1\\\hline
Religiose/Philosophische Ansichten  &
String &
n\\\hline
sonstiges &
String &
n\\\hline
\end{supertabular}
\end{center}

\newpage
\subsection[Bevorzugte Äußerligkeit]{Bevorzugte Äußerligkeit}
\textcolor{zubesprechen}{(gleiche Daten(also Aussehen) müssen auch im Profilen
angegeben werden , sonst kann man nicht nach diesen Kriterien suchen)}

\begin{center}
\tablefirsthead{}
\tablehead{}
\tabletail{}
\tablelasttail{}
\begin{supertabular}{|m{5.467cm}|m{5.467cm}|m{5.467cm}|}
\hline
Feldname &
Type &
Kardinalität\\\hline
\_id &
int &
1\\\hline
Haarfarbe &
String &
1\\\hline
Haarstyle &
String (welig, glatt, ...) &
1\\\hline
Augenfarbe &
String &
1\\\hline
Hautfarbe &
String &
1\\\hline
Körpergrö{\ss}e &
int &
1\\\hline
Körperstatur &
String &
1\\\hline
Tatoos &
boolean &
1\\\hline
\end{supertabular}
\end{center}

\subsection[Freudenliste:]{Freudenliste:}
\textcolor{zubesprechen}{(evtl. direkt ins Profil \ ein Array mit ids von befreundeten Profilen einfügen)}

\begin{center}
\tablefirsthead{}
\tablehead{}
\tabletail{}
\tablelasttail{}
\begin{supertabular}{|m{5.467cm}|m{5.467cm}|m{5.467cm}|}
\hline
Feldname &
Type &
Kardinalität\\\hline
\_id &
int &
1\\\hline
profil\_id\_1 &
int &
1\\\hline
profil\_id\_2 &
int &
n\\\hline
\end{supertabular}
\end{center}

\subsection[Freundschaftsanfragen:]{Freundschaftsanfragen:}
\begin{center}
\tablefirsthead{}
\tablehead{}
\tabletail{}
\tablelasttail{}
\begin{supertabular}{|m{5.467cm}|m{5.467cm}|m{5.467cm}|}
\hline
Feldname &
Type &
Kardinalität\\\hline
\_id &
int &
1\\\hline
user\_profil\_id &
int &
1\\\hline
gefragte\_profil\_id &
int &
1\\\hline
status &
String (bestätigt, abgelehnt, unbekannt) &
1\\\hline
\end{supertabular}
\end{center}
\newpage
\subsection[Liste gemeldeter Profile]{Liste gemeldeter Profile:}
\begin{center}
\tablefirsthead{}
\tablehead{}
\tabletail{}
\tablelasttail{}
\begin{supertabular}{|m{5.467cm}|m{5.467cm}|m{5.467cm}|}
\hline
Feldname &
Type &
Kardinalität\\\hline
\_id &
int &
1\\\hline
gemeldete\_profil\_id &
int &
1\\\hline
Grund  &
String &
1\\\hline
\end{supertabular}
\end{center}


\textcolor{zubesprechen}{(Vorlieben?, Vorstellungen? 8.2 Ansicht: Anmeldung \ )}


\subsection[E-Mail-Box:]{E-Mail-Box:}
\begin{center}
\tablefirsthead{}
\tablehead{}
\tabletail{}
\tablelasttail{}
\begin{supertabular}{|m{5.467cm}|m{5.467cm}|m{5.467cm}|}
\hline
Feldname &
Type &
Kardinalität\\\hline
\_id &
int &
1\\\hline
sender\_id &
int &
1\\\hline
empfänger\_id &
int &
1 (n?)\\\hline
Subject &
String &
1\\\hline
inhat &
String &
1\\\hline
\end{supertabular}
\end{center}

\bigskip
\color{zubesprechen}
Datenschutzeinstellungen

\begin{itemize}
\item Geburtsdatum ansehen (alle/nur freunde/ nur ich)
\item Berechtigung Bilder ansehen (alle/nur freunde/ nur ich)
\item Berechtigung Nachrichten schreiben(alle/nur freunde/ nur admin)
\end{itemize}

{sanktionen (Welche Sanktionen werden Angewendet?, evtl. werden auch im Profil
vermerkt (z.B 3 Tage Nachrichten-Sperre) oder Anzahl \ der Verwarnungen)

? Performanze frage:

Documente so aufteilen wie in rationaen DB:

+ kleinere documente kriegen und parsen

- mehrere Get befehe an db wenn man Fremdschlüsseln hat

mögigst wenig Documente haben (d.h da wo jetzt Id stehen, weren tatsächichen daten stehen z.b in arrays)

+ weniger anfragen an db

- ein Riesendocument aufeinmal kriegen, den man dann parsen muss


\bigskip
\end{document}
